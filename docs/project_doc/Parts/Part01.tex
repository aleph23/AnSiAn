%************************************************
\chapter{Introduction}\label{ch:introduction}
%************************************************
\glsresetall % Resets all acronyms to not used

\ac{AnSiAn} is a project of the Secure Mobile Network Lab (SEEMOO) at the
Technical University in Darmstadt. The Android applications features a
graphical signal analyzer that can be used with common \acp{SDR} like the
HackRF and the RTL-SDR. The project is based on RF Analyzer, an application by
Dennis Mantz. AnSiAn currently adds the following features on top of the set of
RF Analyzer:
\begin{itemize}
	\item Time Domain Signal Graph (Waveform)
	\item Morse Decoder
	\item Scanner
	\item Codebase structured according to the MVC pattern
\end{itemize}

This lab aims to further extend the feature-set of \ac{AnSiAn} while also
making the app more stable and refining existing features. The description of
the project goals are listed in section \ref{sec:project_definition}.


\section{Project Definition}
\label{sec:project_definition}

The project definition section defines the features that will be implemented
and schedules them into three sprints.

\subsection{Features}

The new features are sorted into must-have and nice-to-have. As can be seen in
section \ref{sec:time_schedule}, the third sprint has time reserved for either
the nice-to-have features or for further working on the must-have features and
the documentation.

\subsubsection{Must-Have}

The following features are planned to be implemented in the first and second
sprint in respecting order:
\begin{itemize}
	\item \ac{RDS} \\
		If the user selects the existing wide-band \ac{FM} demodulation option
		the app shall try to detect and demodulate any existing \ac{RDS}
		signal along with the audio demodulation. The extracted information
		(channel name and time) shall be printed on the screen.
	\item PSK31 \\
		If the user selects either of the single side band demodulation modes
		(\ac{USB} and \ac{LSB}) he shall have the option to also enable
		PSK31 demodulation along with or instead of the audio demodulation.
		The demodulated text string should appear and scroll through the
		analyzer window.
	\item Extract RDS-, Morse and PSK31-Text to file \\
		If the user selects to demodulate any digital mode, the demodulated
		text shall be written to a user configured log file.
	\item rad1o support (for receiving) \\
		The rad1o badge, which is a modified low-cost replica of the HackRF
		shall be supported as signal source by AnSiAn.
	\item sending with HackRF and rad1o \\
		If the user connects a \ac{SDR} with transmission capabilities to
		his Android device, he shall have the possibility to transmit signals:
		\begin{itemize}
			\item Replay I/O samples from a file
			\item Generate and send Morse code from text
			\item FM audio modulation from a file
		\end{itemize}
\end{itemize}

\subsubsection{Nice-to-Have}

The nice-to-have features are scheduled in the third and last sprint. However,
they will only be added to the feature-set if the last sprint is not needed
in order to compensate delays on the must-have features. The features are
listed in the order of priority:
\begin{itemize}
	\item Walkie-Talkie Mode \\
		The user shall have the possibility to put AnSiAn into a Walkie-
		Talkie mode. In this mode the analyzer will demodulate an FM channel
		and the user can quickly switch between demodulation and transmission
		of audio from the internal microphone.
	\item Packet Radio \\
		A new mode \emph{Packet Radio} shall be added to the existing
		demodulation modes of AnSiAn. Once selected it will allow the user
		to tune to a Packet Radio channel and see information about 
		demodulated packets on the screen. If time permits it might even
		be possible to implement a transmission feature for Packet Radio.
\end{itemize}


\subsection{Time Schedule}
\label{sec:time_schedule}

The project will have two developers, Max Engelhardt and Dennis Mantz,
working in three sprints. There are three milestones corresponding with
the sprints, labeled alpha, beta and final version. Although the
milestones are labeled according to the SEEMOO lab requirements, they
each add independent and self-contained features to the application:

\begin{itemize}
	\item Sprint 1: alpha version (due 09.06.)
	\begin{itemize}
		\item RDS
		\item PSK31
		\item Extract RDS-, Morse and PSK31-Text to file
	\end{itemize}
\end{itemize}

\begin{itemize}
	\item Sprint 2: beta version (due 21.07.)
	\begin{itemize}
		\item rad1o support (for receiving)
		\item Transmission support for HackRF and rad1o
		\begin{itemize}
			\item replay I/O samples
			\item generate and send Morse code from text
			\item FM audio modulation
		\end{itemize}
	\end{itemize}
\end{itemize}

\begin{itemize}
	\item Sprint 3: final version (due 25.08.)
	\begin{itemize}
		\item Complete leftovers from previous sprints
		\item Walkie-Talkie Mode (optional)
		\item Packet Radio (optional)
	\end{itemize}
\end{itemize}



%%*****************************************
%\chapter{Related Work}\label{ch:relatedwork}
%%*****************************************
%\glsresetall % Resets all acronyms to not used
%
%\lipsum[4]
